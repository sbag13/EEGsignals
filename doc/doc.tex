\documentclass[12pt]{report}

\usepackage{polski}
\usepackage[utf8]{inputenc}
\usepackage[T1]{fontenc}
\usepackage{enumitem}

\usepackage[
backend=biber,
style=alphabetic,
sorting=ynt
]{biblatex}
\addbibresource{doc.bib}

\title{\Huge Zastosowania informatyki w medycynie \\[1cm]\Huge Automatyczne rozpoznawanie faz snu na podstawie diagramów EEG}
\author{Szymon Bagiński \\[1cm]{\small Prowadzący: Dr hab. inż. Robert Burduk}}
% \author{Szymon Bagiński\thanks{funded by the ShareLaTeX team}}
\date{Czerwiec 2018}

\begin{document}

    \begin{titlepage}
        \maketitle
    \end{titlepage}

    \tableofcontents

    \chapter*{Wstęp}
    \addcontentsline{toc}{chapter}{Wstęp}
        Celem projektu było stworzenie programistycznej metody automatycznego rozpoznawania faz snu człowieka na podstawie zapisu sygnału elektroencefalograficznego (\textit{EEG}). Częścią zadania nie było opracowanie metody gromadzenia zapisów ani wykonywanie pomiarów sygnału. Zdecydowano się skorzystać z ogólnodostępnej internetowej bazy danych \cite{}, udostępniającej pliki w formacie \textit{EDF}. % TODO odnoscnik do bazy i do formatu pliku

        Z uwagi na brak wiedzy dziedzinowej, zdecydowano się na zastosowanie sztucznej sieci neuronowej zamiast analitycznego podejścia do problemu. Przy implementacji sieci skorzystano z otwartoźródłowej biblioteki programistycznej \textit{TensorFlow}. Wykorzystano wersję biblioteki wykorzystującą technologię \textit{CUDA}, dzięki czemu skrócono czas uczenia sieci poprzez przeniesienie wielu równoległych obliczeń na kartę graficzną. % TODO odnoscik tensorflow i do CUDA

        Do wykonania skryptów wczytujących i przygotowujących dane, oraz do uczenia maszynowego został użyty język programowania \textit{Python}. Szczególnie przydatne okazały się pakiety:
        \begin{itemize}
            \item NumPy - obsługa danych tablicowych,
            \item PyEDFlib - obsługa plików \textit{EDF},
            \item PyWavelets - przetwarzania sygnału, Dyskretna Transformata Falkowa,
            \item TensorFlow - interfejs biblioteki do języka \textit{Python}.
        \end{itemize}

    \chapter{Pozyskanie danych}

    \chapter{Wyekstrahowanie cech z sygnału EEG}

    \chapter{Klasyfikacja faz snu w oparciu o sieć neuronową}

    \chapter*{Podsumowanie}
    \addcontentsline{toc}{chapter}{Podsumowanie}

    \cite{dirac}    % TODO do usuniecia
    \printbibliography[heading=bibintoc, title={Bibliografia}]

\end{document}