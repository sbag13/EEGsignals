\documentclass[12pt]{report}

\usepackage{geometry}
\usepackage{polski}
\usepackage[utf8]{inputenc}
\usepackage[T1]{fontenc}
\usepackage{enumitem}

\usepackage[
    backend=biber,
    style=alphabetic,
    sorting=ynt
    ]{biblatex}

\addbibresource{doc.bib}
\geometry{legalpaper, margin=1in}

\title{\Huge Zastosowania informatyki w medycynie \\[1cm]\Huge Automatyczne rozpoznawanie faz snu na podstawie diagramów EEG}
\author{Szymon Bagiński \\[1cm]{\small Prowadzący: Dr hab. inż. Robert Burduk}}
% \author{Szymon Bagiński\thanks{funded by the ShareLaTeX team}}
\date{Czerwiec 2018}

\begin{document}

    \begin{titlepage}
        \maketitle
    \end{titlepage}
    
    \tableofcontents

    \chapter*{Wstęp}
    \addcontentsline{toc}{chapter}{Wstęp}
        Celem projektu było stworzenie programistycznej metody automatycznego rozpoznawania faz snu człowieka na podstawie zapisu sygnału elektroencefalograficznego (\textit{EEG}). Częścią zadania nie było opracowanie metody gromadzenia zapisów ani wykonywanie pomiarów sygnału. Zdecydowano się skorzystać z ogólnodostępnej internetowej bazy danych \cite{}, udostępniającej pliki w formacie \textit{EDF} (European Data Format). % TODO odnoscnik do bazy i do formatu pliku

        Z uwagi na brak wiedzy dziedzinowej, przy klasyfikacji zdecydowano się na zastosowanie sztucznej sieci neuronowej zamiast analitycznego podejścia do problemu. Przy implementacji sieci skorzystano z otwartoźródłowej biblioteki programistycznej \textit{TensorFlow}. Wykorzystano wersję biblioteki wykorzystującą technologię \textit{CUDA}, dzięki czemu skrócono czas uczenia sieci poprzez przeniesienie wielu równoległych obliczeń na kartę graficzną. % TODO odnoscik tensorflow i do CUDA

        Do wykonania skryptów wczytujących i przygotowujących dane, oraz do uczenia maszynowego został użyty język programowania \textit{Python}. Szczególnie przydatne okazały się pakiety:
        \begin{itemize}
            \item NumPy - obsługa danych tablicowych,
            \item PyEDFlib - obsługa plików \textit{EDF},
            \item PyWavelets - przetwarzania sygnału, Dyskretna Transformata Falkowa,
            \item TensorFlow - interfejs biblioteki do języka \textit{Python}.
        \end{itemize}

        Wszystkie skrypty, które powstały w ramach projektu można znaleźć w publicznym repozytorium \cite{}. % TODO

    \chapter{Pozyskanie danych}
        Użyte w projekcie dane zostały pobrane ze strony internetowej \cite{}. Wykorzystane zapisy uzyskano w latach 1987-1991 podczas badań nad wpływem wieku na sen. Rekordy pochodzą od osób rasy kaukaskiej w wieku od 25 do 101 lat , które nie przyjmowały żadnych leków związanych ze snem. Sygnały były próbkowane z częstotliwością 100 Hz, a ich klasyfikacja została przeprowadzona przy pomocy reguł Rechtschaffen'a i Kales'a na bazie trzydziestosekundowych segmentów z kanałów \textit{Fpz-Cz} i \textit{Pz-Oz}. Każdy z segmentów został sklasyfikowany jako: przebudzony, nie-REM1, nie-REM2, nie-REM3, REM, niewielka aktywność, poruszenie lub nieokreślony.
        W tym projekcie został użyty jednak tylko kanał \textit{Pz-Oz}, a rozpoznawanymi fazami były: przebudzony, nie-REM1 + REM, nie-REM2, nie-REM3 + niewielka aktywność.
        Dokładny opis przebiegu badań znajduje się stronie internetowej bazy danych.
        
        Do trenowania sztucznej sieci neuronowej zostało wykorzystanych dwadzieścia siedem ponad dwudziestogodzinnych zapisów \textit{EEG}, a skuteczność klasyfikacji sprawdzono przy pomocy pięciu rekordów nie użytych przy trenowaniu. Lista plików znajduje się w dodatku % TODO appendix

        \section{Wczytywanie danych z pliku EDF}
            W module \textit{EdfFile.py} została stworzona klasa \textit{EdfFile}, która jako argument konstruktora przyjmuje ścieżkę do pliku z elektroencefalogramem oraz ścieżkę do pliku, w którym znajdują się adnotacje.
            
            W funkcji \textit{\_\_init\_\_} zostało wykorzystanych kilka funkcjonalności z biblioteki PyEDFlib. Funkcja \textit{EdfReader} służy do odczytania danych z pliku i zwraca obiekt go reprezentujący, przez który mamy dostęp do właściwości zapisu takich jak między innymi:
            \begin{itemize}
                \item liczba sygnałów (kanałów) w pliku,
                \item lista etykiet sygnałów,
                \item wartości napięcia dla zczytanych próbek,
                \item lista częstotliwości sygnałów,
                \item lista adnotacji do faz snu,
                \item lista czasów przejść między fazami snu.                
            \end{itemize}
            W tym miejscu jest tworzona także kolekcja obiektów typu \textit{Signal}, które są "opakowaniem" na sygnały obecne w pliku.

            W klasie \textit{EdfFile} znajdują się także dwie funkcje odpowiedzialne za przygotowanie danych dla sieci neuronowej. Są to \textit{createOutput} oraz \textit{createInput}. Zwracają one dane odczytane z pliku w postaci macierzy, które następnie w skrypcie \textit{prepare.py} są agregowane. Tutaj także, ma miejsce kontrola poprawności danych i usunięcie błędnych próbek, aby nie umieszczać ich na wejściu sieci.

        \section{Przetwarzanie wstępne}
                        

    \chapter{Wyekstrahowanie cech z sygnału EEG}

    \chapter{Klasyfikacja faz snu w oparciu o sieć neuronową}

    \chapter*{Podsumowanie}
    \addcontentsline{toc}{chapter}{Podsumowanie}

    \cite{dirac}    % TODO do usuniecia
    \printbibliography[heading=bibintoc, title={Bibliografia}]

\end{document}